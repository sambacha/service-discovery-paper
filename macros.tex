%!TEX root = main.tex
%=========================================================

% use true/false to toggle all comments (both kinds)

\newboolean{showcomments}
\setboolean{showcomments}{true}

% ====== comments ======
\newcommand\important[1]{\todo[inline]{\textbf{Important:} #1}}
\newcommand\alberto[1]{\todo[color=yellow,inline]{\textbf{Alberto:} #1}}
\newcommand\etienne[1]{\todo[color=orange,inline]{\textbf{Etienne:} #1}}
\newcommand\pierre[1]{\todo[color=brown,inline]{\textbf{Pierre:} #1}}
\newcommand\felix[1]{\todo[color=blue!40,inline]{\textbf{Felix:} #1}}
\newcommand\michal[1]{\todo[color=green,inline]{\textbf{Michał:} #1}}
\newcommand\onur[1]{\todo[color=red,inline]{\textbf{Onur:} #1}}
\newcommand\sergi[1]{\todo[color=pink,inline]{\textbf{Sergi:} #1}}
\newcommand\ramin[1]{\todo[color=brown,inline]{\textbf{Ramin:} #1}}
\newcommand\mustafa[1]{\todo[color=brown,inline]{\textbf{Mustafa:} #1}}
% Uncomment the following command to make all comments disappear
\ifthenelse{\boolean{showcomments}} { }
{
\renewcommand\important[1]{}
\renewcommand\alberto[1]{}
\renewcommand\mustafa[1]{}
\renewcommand\michal[1]{}
\renewcommand\onur[1]{}
\renewcommand\sergi[1]{}
\renewcommand\etienne[1]{}
\renewcommand\pierre[1]{}
\renewcommand\ramin[1]{}
\renewcommand\felix[1]{}
}

% ====== inlined and toggable comments ======

\ifthenelse{\boolean{showcomments}}
{ \newcommand{\mynote}[3]{
    \protect\fbox{\bfseries\sffamily\scriptsize#1}
    {\small\textsf{\emph{\color{#3}{#2}}}}}}
{ \newcommand{\mynote}[3]{}}

\newcommand{\er}[1]{\mynote{Etienne}{#1}{blue}}
\newcommand{\mk}[1]{\mynote{Michał}{#1}{brown}}
\newcommand{\sr}[1]{\mynote{Sergi}{#1}{red}}

% \newcommand{\xxx}[1]{\mynote{YourName}{#1}{black!20!red!80!}}
% \newcommand{\xxx}[1]{\mynote{YourName}{#1}{green}}
\newcommand{\as}[1]{\mynote{Alberto}{#1}{orange}}
\newcommand{\dk}[1]{\mynote{Daniel}{#1}{green}}
\newcommand{\rs}[1]{\mynote{Ramin}{#1}{violet}}
\newcommand{\oa}[1]{\mynote{Onur}{#1}{red}}

% ====== systems ======
\newcommand\sysname{TOPDISC\xspace}
\newcommand\discv{DISCv4\xspace}
\newcommand\altname{DHT\xspace}
\newcommand\altnameticket{DHTTicket\xspace}
\newcommand\altsysname{PMETIS\xspace}
\newcommand\sysnamePriv{Pineapple\xspace}
\newcommand\sysnameAnon{\coconut}
\newcommand\libcoin{LibraCoin\xspace}
\newcommand\libcoins{LibraCoins\xspace}
\newcommand\privcoin{PrivCoin\xspace}
\newcommand\privcoins{PrivCoins\xspace}

\newcommand\sysnamereplay{\texttt{byzcuit}\xspace}
\newcommand\sysnamebaseline{\texttt{byzcuit-baseline}\xspace}
\newcommand\simplesysname{Simple-\sysname}

\newcommand\libra{Libra\xspace}
\newcommand\fourier{Fourier\xspace}
\newcommand\chainspace{Chainspace\xspace}
\newcommand\ethereum{Ethereum\xspace}
\newcommand\hyperledger{Hyperledger\xspace}
\newcommand\omniledger{Omniledger\xspace}
\newcommand\rapidchain{RapidChain\xspace}
\newcommand\elgamal{El-Gamal\xspace}
\newcommand\coconut{Coconut\xspace}
\newcommand\macggm{$\bm{\mathrm{MAC_{GGM}}}$\xspace}
\newcommand\sbac{S-BAC\xspace}
\newcommand\bft{BFT\xspace}
\newcommand\atomix{Atomix\xspace}
\newcommand\cscoin{CSCoin\xspace}
\newcommand\rscoin{RSCoin\xspace}
\newcommand\lsbac{\sysname}
\newcommand\fsbac{F-SBAC\xspace}
\newcommand\bftsmart{\textsc{bft-SMaRt}\xspace}

\makeatletter
\def\BState{\State\hskip-\ALG@thistlm}
\makeatother

% ====== custom notations ======
%\newcommand\algorithm[1]{\textsf{#1}}
\newcommand\hashtopoint{H^*\xspace}
\newcommand\stringtopoint{H'\xspace}
\newcommand\function[1]{\ding{118}\xspace \textsf{#1}:\xspace}
\newcommand\shard[1]{\emph{shard}\xspace#1\xspace}
\newcommand\Shard[1]{\emph{Shard}\xspace#1\xspace}
\newcommand\preacceptt{\textsf{pre-accept}($T$)\xspace}
\newcommand\preabortt{\textsf{pre-abort}($T$)\xspace}
\newcommand\preaccepttt{\textsf{pre-accept}($T'$)\xspace}
\newcommand\preaborttt{\textsf{pre-abort}($T'$)\xspace}
\newcommand\preacceptttt{\textsf{pre-accept}($T''$)\xspace}
\newcommand\preacceptts{\textsf{pre-accept}($T,s_T$)\xspace}
\newcommand\preabortts{\textsf{pre-abort}($T,s_T$)\xspace}
\newcommand\preabortttt{\textsf{pre-abort}($T''$)\xspace}
\newcommand\acceptt{\textsf{accept}($T$)\xspace}
\newcommand\abortt{\textsf{abort}($T$)\xspace}
\newcommand\accepttt{\textsf{accept}($T'$)\xspace}
\newcommand\aborttt{\textsf{abort}($T'$)\xspace}
\newcommand\acceptttt{\textsf{accept}($T''$)\xspace}
\newcommand\abortttt{\textsf{abort}($T''$)\xspace}
\newcommand\acceptts{\textsf{accept}($T,s_T$)\xspace}
\newcommand\abortts{\textsf{abort}($T,s_T$)\xspace}
\newcommand\myrow[1]{row\xspace\textsf{#1}\xspace}
\newcommand\mycolumn[1]{column\xspace\textsf{#1}\xspace}
\newcommand\shardled{shard-led\xspace}
\newcommand\Shardled{Shard-led\xspace}
\newcommand\clientled{client-led\xspace}
\newcommand\Clientled{Client-led\xspace}
\newcommand\activeObj{`active'\xspace}
\newcommand\inactiveObj{`inactive'\xspace}
\newcommand\locked{`locked'\xspace}
\newcommand\wa{{WA}\xspace}
\newcommand\wb{{WB}\xspace}
\newcommand\ka{{KA}\xspace}
\newcommand\kb{{KB}\xspace}

% ====== concepts/terminology =======

\newcommand\attacker{attacker\xspace}
\newcommand\adversary{attacker\xspace}
\newcommand\prerecorded{prerecorded\xspace}
\newcommand\prerecords{prerecords\xspace}
\newcommand\prerecord{prerecord\xspace}

%  ===== formatting ======
% Abbreviations etc.
\newcommand{\cf}{cf.\@\xspace}
\newcommand{\vs}{vs.\@\xspace}
\newcommand{\etc}{etc.\@\xspace}
\newcommand{\ala}{ala\@\xspace}
\newcommand{\wrt}{w.r.t.\@\xspace}
\newcommand{\etal}{\textit{et al.}\@\xspace}
\newcommand{\eg}{\textit{e.g.}\@\xspace}
\newcommand{\Eg}{\textit{E.g.}\@\xspace}
\newcommand{\ie}{\textit{i.e.}\@\xspace}
\newcommand{\Ie}{\textit{I.e.}\@\xspace}
\newcommand{\via}{\textit{via}\@\xspace}
\newcommand{\defacto}{\textit{de facto}\@\xspace}

\newcommand\mypara[1]{\vspace{0.05in} \noindent \textbf{#1.}}
\newcommand\para[1]{\vspace{0.05in} \noindent \textbf{#1.}}


\newcommand\definition[2]{\ding{118}\xspace \textsf{#1}\xspace$\bm{\rightarrow}$\xspace(#2):\xspace}



% For inlined section titles.
\newcommand\inlinesection[1]{{\bf #1.}}

\def\first{({\it i})\xspace}
\def\second{({\it ii})\xspace}
\def\third{({\it iii})\xspace}
\def\fourth{({\it iv})\xspace}
\def\fifth{({\it v})\xspace}
\def\sixth{({\it vi})\xspace}

\newcommand{\one}{({\em i})\xspace}
\newcommand{\two}{({\em ii})\xspace}
\newcommand{\three}{({\em iii})\xspace}
\newcommand{\four}{({\em iv})\xspace}
\newcommand{\five}{({\em v})\xspace}

% Colours
\definecolor{verylightgray}{gray}{0.8}

% Table
\newcolumntype{L}{l<{\hspace{1cm}}}
\newcolumntype{C}{c<{\hspace{1cm}}}
\newcolumntype{D}{c<{\hspace{0.3cm}}}

% Markers
\newcommand\vgap{\vskip 2ex}
\newcommand\marker{\vgap\ding{118}\xspace}

\def\na{--}
\def\unsure{?}
\def\missing{$!$}
\newcommand{\yes}{\ding{51}}
\newcommand{\no}{\ding{55}}
\DeclareRobustCommand\pie[1]{
\tikz[every node/.style={inner sep=0,outer sep=0, scale=1.5}]{
\node[minimum size=1.5ex] at (0,-1.5ex) {}; 
 \draw[fill=white] (0,-1.5ex) circle (0.75ex); \draw[fill=black] (0.75ex,-1.5ex) arc (0:#1:0.75ex); 
}
}
\def\L{\pie{0}} % Low
\def\M{\pie{-180}} % Medium
\def\H{\pie{360}} % High

