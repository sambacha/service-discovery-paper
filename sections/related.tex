%!TEX root = ../main.tex
%=========================================================

\section{Related Work}

"Eclipsing Ethereum Peers with False Friends"~\cite{henningsen2019eclipsing} - 

"Ethereum eclipse attacks"~\cite{wust2016ethereum} - 

"Low-Resource Eclipse Attacks on Ethereum's Peer-to-Peer Network."~\cite{marcus2018low} - 

"Sybil-resistant DHT routing"~\cite{danezis2005sybil}

"Whanau: A sybil-proof distributed hash table"~\cite{lesniewski2010whanau} - 

"Sybilinfer: Detecting sybil nodes using social networks."~\cite{danezis2009sybilinfer}

"Design and evaluation of Persea, a Sybil-resistant DHT"~\cite{al2014design} - 

"Defending the sybil attack in p2p networks: Taxonomy, challenges, and a proposal for self-registration"~\cite{dinger2006defending}

"Persea: a sybil-resistant social dht"~\cite{al2013persea} - 

"Quantitative analysis of the sybil attack and effective sybil resistance in peer-to-peer systems"~\cite{jetter2010quantitative}

"A Sybil-proof one-hop DHT"~\cite{lesniewski2008sybil}

"Efficient DHT attack mitigation through peers' ID distribution"~\cite{cholez2010efficient}

"Sloppy hashing and self-organizing clusters"~\cite{freedman2003sloppy}

"S-Kademlia"~\cite{pecori2016s}

"Centralized and distributed protocols for tracker-based dynamic swarm management"~\cite{dan2012centralized}

"Supporting k-nearest service discoveries for large-scale edge computing environments"~\cite{teranishi2018supporting}

"Endorsement in Hyperledger Fabric via service discovery"~\cite{manevich2019endorsement}: allows Hyperledger Fabric client to locate available services (chaincodes) using an API since HLF version 1.2. Before the set of services (chaincodes) was hardcoded at the client and server side. Since HLF is a smaller scale private blockchain it does not require large-scale service discovery as ours and it does not need to prevent this service discovery feature from attackers.

"Decentralized identifiers for peer-to-peer service discovery"~\cite{farmer2021decentralized}: besides the service discovery feature in Ethereum itself, some applications build service discovery over Ethereum, as in this example of decentralized identifiers (there are tons of examples of web services using the blockchain to store and retrieve service representatives)~\cite{keizer2021flock}.

"Under the Hood of the Ethereum Gossip Protocol"~\cite{kiffer2021under}: a study of Ethereum gossip protocol that I did not read yet.
Ethna~\cite{wang2021ethna} seems also similar.

TERA~\cite{baldoni2007tera}: A topic-based pub/sub system based on gossip protocols and self organization. Each group of nodes registered for the same topic form a random graph in which peer sampling allows contacting random nodes and use a gossip-based dissemination protocol. A similarity to our work is that nodes advertise their dissemination group with a frequency that depends on the relative size of that group compared to other groups. This relies on gossip-based size estimation protocols. Peers in large groups advertise less often. Registrars keeps the $k$ most recent ads received. The goal is that all groups are equally represented and likely to be found through a random walk. In contrast with our work, TERA requires that all nodes be trusted for not advertising themselves too often.
