%!TEX root = ../main.tex
%=========================================================

\section{Related Work}
\label{sec:related}

\mk{TODO:Mention previous work suggested by the NDSS reviewers~\cite{liu2010netfence, crowcroft2007lazy, kung2015practical}} \er{as per discussion, these should be discussed in the RW and not in the admission section.}

A decentralized service discovery system can be organized by directly storing the membership information in a DHT~\cite{can,chord,rowstron2001pastry,maymounkov2002kademlia}. 
DHT-based solutions offer fault-tolerant, scalable and efficient ways of finding nodes in large-scale networks.
However, it is difficult to guarantee the availability of published service descriptions.
If nodes close to a service hash fail, the whole sub-network becomes undiscoverable. 
While solutions such as Chord4S~\cite{chord4s} reduce this risk, the main drawback remains the vulnerability to Sybil attacks.

Other systems implement service discovery on top of publish-subscribe platforms.
However, those solutions are built directly on top of a DHT~\cite{scribe,poldercast,banno2015} (and share its weaknesses), introduce significant overhead to keep the data up to date~\cite{gossipsub}, introduce a single point of failure~\cite{dan2012centralized}, or require all nodes to be correct (\ie not byzantine)~\cite{baldoni2007tera}.

Recently, multiple service discovery protocols were implemented for the blockchain space~\cite{farmer2021decentralized, manevich2019endorsement, keizer2021flock}.
Unfortunately, these solutions are meant to work in small-scale systems~\cite{farmer2021decentralized}, or require writing to the blockchain (thus introducing significant monetary and/or computational cost)~\cite{manevich2019endorsement, keizer2021flock}. 

Multiple works proposed DHT enhancements to make it more resistant to Sybil attacks.
This can be achieved by exploiting social relations between participants operating the nodes~\cite{danezis2005sybil, danezis2009sybilinfer}, introducing some kind of Proof-of-Work~\cite{skad} or sampling participant identifiers~\cite{cholez2010efficient}.
All these solutions are difficult to implement in current P2P networks, and may have a negative impact on privacy.
An extensive number of systems have been proposed for resilient peer sampling in P2P networks~\cite{bortnikov2009brahms, jelasity2007gossip, ouguz2014stable, pigaglio2022raptee}.
While those systems are useful in some scenarios, they cannot be easily adapted to application-specific peer sampling required by the Ethereum ecosystem.

Relevant to our work, the Ethereum DHT was recently enhanced~\cite{marcus2018low, henningsen2019eclipsing} to make it more resistant to low-resource eclipse attacks at the DHT level. Those solution enable \sysname to operate, as it relies on honest participants not being fully eclipsed at the DHT level.

The enforcement of a waiting time for incoming requests to improve fairness and implement rate control has been used for preventing DDoS attacks towards centralized services or domains.
In this context, the key interest is to avoid maintaining per-client state at a server while offering priority services to clients that have waited the longest, and to adjust waiting times based on per-domain traffic and congestion.
Implementations of this idea include NetFence~\cite{liu2010netfence}, Lazy Suzan~\cite{crowcroft2007lazy} and the work of Kung \emph{et al.}~\cite{kung2015practical}.
In contrast with \sysname, however, these approaches do not focus on content (topic ads) diversity, but only use enforced waiting for rate control.
% They also do not consider a decentralized system.

%\er{Maybe discuss the byzantine-resilient peer sampling system Brahms, that also uses a blocking mechanism when more than a threshold number of view updates (pushes) are received in a certain amount of time (not sure if this is overall or for a specific node though)~\cite{}}

%There exists several protocols aimed at discovery devices providing network services for local area networks.
%For instance, the Simple Service Discovery Protocol (SSDP)~\cite{helal2002standards}, basis of the discovery protocol of Universal Plug and Play (UPnP)~\hl{missing ref}, or the DNS-based service discovery~\cite{RFC6763}, are used to advertise services that other devices provide, such as printers, webcams, HTTPS servers, and other mobile devices, usually using multicasting or broadcasting techniques.
%However for service discovery in Internet-wide environments, since it is not possible to broadcast, centralized setups are commonly used to provide service discovery with good performance.
%UDDI~\cite{zhang2002aggregate} has been recognized as the most popular discovery mechanism for Web services.

%The attack ensures that the lookup-buffer used to initiate outbound connections is filled up with adversarial nodes by placing an adversarial node to each one of the DHT buckets, requiring only 2 IPs from distinct /24 subnets to be successful.
%As a result of this paper,  Ethereum network and Geth v1.8 and v1.9 implemented new countermeasures, such as  increasing number of connections, considering all nodes of the table during lookups, or throttling the inbound connection attempt, to reduce the chance of selecting an attacker-node.
%These countermeasure have been added in our simulations for the underlying Kademlia protocol.

%In the area of avoiding Sybil and derived eclipsing attacks in peer-to-peer networks several solutions and state-of-the-art can be found in the literature.
%In~\cite{danezis2005sybil}, different strategies are devised to make DHTs resilient to malicious nodes trying to poison lookups, by routing queries in a way that minimizes trust bottlenecks, to minimize the amount of poisoned information that honest nodes receive from adversarial nodes.
%In a similar way, S/Kademlia~\cite{skad} propose new mechanisms to get resilience against common attacks by using parallel lookups over multiple disjoint paths, limiting free nodeld generation with crypto puzzles and introducing a reliable sibling broadcast.
%In ~\cite{cholez2010efficient}, the authors propose a statistical approach to detect a particular type of Sybil attack in  Kademlia DHT, where Sybil peers strategically choose IDs that are close to a target ID in the DHT ID space.

%The authors found that the expected the ID distribution of the closest nodes returned in the search results for target IDs follow a geometric distribution. Therefore, the divergence from geometric distribution of the node IDs in search results indicate existence of Sybil nodes in the results. However, computing a divergence threshold is not straightforward and requires fine tuning to avoid false positives when detecting Sybil nodes.


%Another possible solution for service discovery in blockchain platforms is to incorporate a service discovery component to the blockchain. 
%In~\cite{manevich2019endorsement}, a new service discovery component is presented for Hyperledger Fabric (HLF), a permissioned blockchain platform.
%The service discovery component provides APIs that allow the client application to dynamically discover the configuration details of the endorsement policies and chaincode it needs to use.
%However, since HLF is a smaller scale private blockchain it does not require large-scale service discovery as ours and it does not need to prevent this service discovery feature from attackers.

%"Endorsement in Hyperledger Fabric via service discovery"~\cite{manevich2019endorsement}: allows Hyperledger Fabric client to locate available services (chaincodes) using an API since HLF version 1.2. Before the set of services (chaincodes) was hardcoded at the client and server side. 
%Similarly, in~\cite{farmer2021decentralized} the authors propose decentralized identifiers for peer-to-peer service discovery.
%"Decentralized identifiers for peer-to-peer service discovery"~\cite{farmer2021decentralized}:
%Besides the service discovery feature in Ethereum itself, some applications build service discovery over Ethereum, as in this example of decentralized identifiers (there are tons of examples of web services using the blockchain to store and retrieve service representatives)~\cite{keizer2021flock}.


%"Centralized and distributed protocols for tracker-based dynamic swarm management"~\cite{dan2012centralized}

%In~\cite{ring}, the authors propose a  universal overlay to provide a scalable infrastructure to bootstrap multiple service overlays providing different functionality, \ie a p2p network to help bootstrapping any p2p network.
%The solution is based on pastry,  but could work with any structured p2p.  The solution assumes node ids are assigned by certification authority,  therefore its security depends on a centralised system.  It utilises three components:
%First, a file storage based on PAST~\cite{past}, a content storage solution using pastry, where content is stored by K users with nodeid closest to content id using the same namespace. It can be used to store p2p software necessary to join a specific network and  signed by the creator. 
%Second, a multicast tree using Scribe~\cite{scribe} to creates multicast groups on top of Pastry. Nodes with closest id to groupid are used as rendezvous points.  Join messages are routed towards groupid and crossed nodes join the multicast tree. Messages are delivered to rendezvous points and follows the distribution tree. 
%Third, a search engine that allows nodes to search for keys using a set of query keywords. An index associates a keyword with a list of service keys, stored to the closest node to the keyword hash. Multicast groups can be used for persistent service querys using specific keywords. When new services are created users subscribed receive the advertisement.
%Advertising and discovery is done by generating a service certificate by the creator that describes the service. Service certificate is stored using a service key. Keywords are associated with a service key for the search engine.
%Services can be discovered either finding the service certificate looking for the service key, or by subscribing to a multicast tree that matches a certain keyword. The service certificate includes service description plus code keys, that can be used to retrieve the software similarly.
%The solution can be very efficient in terms of routing and network load. However, in terms of security it generates lots of doubts. Not clear to me if there can be multiple creators for the same service, and how malicious nodes are avoided to create fake services. I think most of the security resides on the certification authority that issues nodeids certificates, and therefore its a sort of permissioned p2p network. 

%Another possible solution for service discovery in blockchain platforms is to incorporate a service discovery component to the blockchain. 
%In~\cite{manevich2019endorsement}, a new service discovery component is presented for Hyperledger Fabric (HLF), a permissioned blockchain platform.
%The service discovery component provides APIs that allow the client application to dynamically discover the configuration details of the endorsement policies and chaincode it needs to use.
%However, since HLF is a smaller scale private blockchain it does not require large-scale service discovery as ours and it does not need to prevent this service discovery feature from attackers.

%"Endorsement in Hyperledger Fabric via service discovery"~\cite{manevich2019endorsement}: allows Hyperledger Fabric client to locate available services (chaincodes) using an API since HLF version 1.2. Before the set of services (chaincodes) was hardcoded at the client and server side. 
%Similarly, in~\cite{farmer2021decentralized} the authors propose decentralized identifiers for peer-to-peer service discovery.
%"Decentralized identifiers for peer-to-peer service discovery"~\cite{farmer2021decentralized}:
%Besides the service discovery feature in Ethereum itself, some applications build service discovery over Ethereum, as in this example of decentralized identifiers (there are tons of examples of web services using the blockchain to store and retrieve service representatives)~\cite{keizer2021flock}.

%"Under the Hood of the Ethereum Gossip Protocol"~\cite{kiffer2021under}: a study of Ethereum gossip protocol that I did not read yet.

%Ethna~\cite{wang2021ethna} seems also similar.




%\subsection{Eclipse attacks in Eth/p2p}

%Approaches to leverage security in DHT-based networks.





\iffalse

%"Efficient DHT attack mitigation through peers' ID distribution"~\cite{cholez2010efficient} – This paper proposes a statistical approach to detect a particular type of Sybil attack in vanilla Kademlia DHT, where Sybil peers strategically choose IDs that are close to a target ID in the DHT ID space. If sufficient number of (i.e., typically 10 or more) Sybil peers are successfully placed closest to a target ID, then the Sybils could attract all or most of the search and registration requests for that ID because of their proximity to that target and launch attacks such as returning bogus search results. On the other hand, the normal behaviour of honest peers is to generate their IDs uniformly at random. Based on measurements on a DHT with only honest peers, the authors find that the expected the ID distribution of the closest nodes returned in the search results for target IDs follow a geometric distribution. Therefore, the divergence from geometric distribution of the node IDs in search results indicate existence of Sybil nodes in the results. Once divergence is detected, the IDs that contribute the most to the divergence are considered to be Sybils and are therefore omitted from the search results. However, computing a divergence threshold is not straightforward and requires fine tuning to avoid false positives when detecting Sybil nodes.


%S/Kademlia~\cite{skad}.
%"S-Kademlia"~\cite{pecori2016s}


Security lessons learned from literature:
\begin{itemize}
\item Assume that the underlying network layer does not provide any security properties to the overlay layer.
\item Importance of difficulty on generating random ids.
\item Nodes should not be capable of generating any id and duplicate ids should not be possible. Ids should be linked to ip, port or public key.
\item Use of parallel lookups over multiple disjoint paths to avoid querying only  adversarial nodes paths.
\item Importance of limiting IPs per bucket to require more resources to launch a sybil attack.
\item Avoid querying only nodes close to topic id / node id because adversarial nodes can strategically place nodes close to those ids.
\end{itemize}

%In "Eclipsing Ethereum Peers with False Friends"~\cite{henningsen2019eclipsing} - ,  the authors present  a false friends attack,  an eclipse attack applicable to Geth version v1.8.20.  The attack ensures that the lookup-buffer used to initiate outbound connections is filled up with adversarial nodes by placing an adversarial node to each one of the DHT buckets. 
%Since there is a limit that at most 2 nodes from the same /24 subnet can be included in the same bucket and at most 10 nodes from the same /24 can be in the whole table,  it requires  2 IPs from distinct /24 subnets to be successful,  and in contrast
%with previous attacks, it can be successfully mounted without
%assuming that the victim node reboots at some point, and can be completed in a matter of days.
%In response to the attack presented in the paper,  Geth version v1.9.0 implemented new countermeasures,  such as i) increasing number of connections from 25 to 50 ii) considering all nodes of the table during lookups, instead of only the bucket heads,  to reduce the chance
%of selection an attacker-node and iii) throttle the inbound connection attempts to limit the consecutive inbound connection attempts from the same IP to 30 seconds.

%"Low-Resource Eclipse Attacks on Ethereum's Peer-to-Peer Network."~\cite{marcus2018low} - 

"Sybil-resistant DHT routing"~\cite{danezis2005sybil} - they enhance standard DHT routing with information about the social network (by whom the nodes where introduced into the DHT). Based on that, they try to detect and avoid Sybils. Again, we don't have an introduction social network. 

\sergi{I haven't read these papers yet. To be included}
"A Sybil-proof one-hop DHT"~\cite{lesniewski2008sybil}



"Sybilinfer: Detecting sybil nodes using social networks."~\cite{danezis2009sybilinfer}
SybilInfer is an algorithm aimed at detecting Sybil attacks in social networks using e Bayesian inference approach.  It  labels which nodes are honest and which are dishonest with a degree of certainty. The decision is based on an analysis of social connections. However, it requires a social network that we do not have in our setup. 

"Whanau: A sybil-proof distributed hash table"~\cite{lesniewski2010whanau} - 


"Persea: a sybil-resistant social dht"~\cite{al2013persea} - 


"Design and evaluation of Persea, a Sybil-resistant DHT"~\cite{al2014design} - 

"Defending the sybil attack in p2p networks: Taxonomy, challenges, and a proposal for self-registration"~\cite{dinger2006defending}


"Quantitative analysis of the sybil attack and effective sybil resistance in peer-to-peer systems"~\cite{jetter2010quantitative}


GossipSub: Attack-Resilient Message Propagation in
the Filecoin and ETH2.0 Networks

\href{https://asc.di.fct.unl.pt/~jleitao/pdf/dinps22-tr.pdf}{Enriching Kademlia by Partitioning} (work in progress report, to be presented at \href{DINPS2022 by Protocol Labs}{https://research.protocol.ai/sites/dinps/calls/})
\fi
