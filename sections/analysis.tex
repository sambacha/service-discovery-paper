\section{Analysis}\label{sec:analysis}
In this section, we analyze the properties of \sysname and choose optimal values for system parameters. Due to space constraints, we include an extended version of the analysis in \Cref{sec:appendix}.
\subsection{Efficiency}

\para{Memory usage is bounded by the capacity of the ad cache} 
We focus uniquely on registrars, as advertisers and searchers require a fixed amount of memory for their operations. The amount of ads in the cache is given by $d = xa/(a + w(x))$, where $x$ is the number of requests constantly trying to get into the table, $a$ is ad lifetime and $w(x)$ is the average waiting time received by requests $x$.
In the worst case scenario, when requests $x$ are able to achieve 0 similarity score for both the topic and the IP addresses, the waiting time formula is given by: $w(x) = ba/(1 - \frac{d}{n})^{P_\textit{occupancy}}$.

The possibility of the cache going above the capacity is determined by the constant $b$ and the exponent $P_\textit{occupancy}$. $b$ should be set to a small value to limit its influence on the waiting time (where IP and topic similarity scores should play the dominant role). $P_\textit{occupancy}$ should be large enough to prevent overflowing of the cache and small enough to enable usage of large portions of the cache under normal traffic conditions. 

In consultation with Geth developers, we assume the capacity of the cache $n = 1000$ and an average size of an advertisement equal to 1KB. \Cref{fig:cache_size_limit} presents the cache usage for different rates of incoming requests. We chose $b=10^{-7}$ and $P_\textit{occupancy}=10$ that provide protections against cache overflowing and good usage of cache space under normal conditions. 

\begin{figure}[t]
    \includegraphics[width=1\linewidth]{img/cache_size_limit}
    % \vspace{-2mm}
    \caption{Ad cache space usage for different request rates.
    }
    \label{fig:cache_size_limit}
\end{figure}
Pending requests (\ie not in the cache) do not create any state (apart from updating the lower bound) at the registrar (\ie the registrar uniquely calculates the waiting time and returns a signed ticket). The lower bound state created by registrars is bounded by the number of distinct IPs and topics in the cash and is thus bounded by its capacity $O(n)$.

\para{Register and lookup operations finish within $O(log(N))$ steps}
Despite some nondeterministic behavior, both operations tightly follow the number of steps of the regular DHT FIND\_NODE operations and thus finish within $O(log(N))$ steps.

\subsection{Fairness}
We assume a Zipf distribution of the topics in the system and that the topic hashes are uniformly distributed in the DHT hash space. For simplicity we uniquely compare the load of \emph{registrar A} - located close to the most popular \emph{topic A} and the load of \emph{registrar B} - located close to the least popular \emph{topic B}. \emph{topic A} is followed by $N_a$ nodes and \emph{topic B} is followed by $N_b$ nodes. 

We assume \emph{topic A} and \emph{topic B} are located on the opposite sides of the DHT hash space (the worst-case scenario for load balancing). Both \emph{registrar A} and \emph{registrar B} receive different amounts of traffic for both \emph{topics A} and {B} ($L_a$ and $L_b$), and the same amount of traffic $L_x$ from other topics \emph{topic X}, $\textit{topic X} \neq \textit{topic A} \land  \textit{topic X} \neq \textit{topic B}$. 

\para{Registration operations achieve equal load distribution}
As the closest node to the \emph{topic A} hash, \emph{registrar A} receives registration requests from all the $N_a$ advertisers. As the furthest node from the \emph{topic B} hash, it also receives $\frac{(N_{b}K_\textit{register})}{(\frac{N}{2})}$ requests for \emph{topic B}. Analogically, \emph{registrar B} receives $N_b$ requests for \emph{topic B} and $\frac{(N_{a}K_\textit{register})}{(\frac{N}{2})}$ requests for \emph{topic A}. As $N_a \gg N_b$, the initial number of request is higher for \emph{registrar A}. However, as its ad cache fills up, \emph{registrar A} will issue higher waiting times making the requests less frequent. \Cref{fig:fairness_registration} presents the registration load ratio between both registrars as a function of increasing popularity between the two topics. The load difference experiences sub-linear growth. When \emph{topic A} is 100 times more popular, \emph{registrar A} receives only 1.6 times more requests than \emph{registrar B}. We present results without the admission control (\ie all the requests receive a fixed waiting time) for reference. 

\begin{figure}[t]
    \includegraphics[width=1\linewidth]{img/fairness_registration}
%    % \vspace{-2mm}
    \caption{Load ratio between registrars located close to the most popular (\emph{registrar A}) and the least popular (\emph{registrar A}) topics.
    }
    \label{fig:fairness_registration}
\end{figure}

\para{Lookup operations achieve equal load distribution}
Let us assume again \emph{registrar A} is the closest node to \emph{topic A}. All the \emph{topic A} searchers $N_a$ will go towards this node during their lookup operations, so the number of requests is expected to grow as the number of $N_a$ grows. At the same time, the more participants in \emph{topic A} the more ads will be placed in other buckets further away from the \emph{registrar A}. Recall that searchers stop their lookup operations after collecting $N_{lookup}$ peers. We set $N_{lookup} = 30$ - a value commonly used by applications in the Ethereum ecosystem. As the number of ads in the network grows, more searchers are likely to stop before reaching \emph{registrar A}. \Cref{fig:fairness_lookup} presents \emph{Registrar A}'s load for increasing $N_a$. Depending on the $K_{register}$ and $K_{lookup}$ parameters, the maximum load is experienced when the number of \emph{topic A} advertisers/searchers is relatively small and goes back to $0$, when the topic becomes popular in the network. We choose $K_{register}$ and $K_register =5$ as a reasonable middle-ground between efficiency, load balance and security.

\begin{figure}[t]
    \includegraphics[width=1\linewidth]{img/fairness_lookup}
%    % \vspace{-2mm}
    \caption{\emph{Registrar A}'s lookup load as a function increasing popularity of \emph{topic A}.
    }
    \label{fig:fairness_lookup}
\end{figure}

\subsection{Security}

\para{\sysname achieves high resistance against eclipse attacks.}
We assume an attacker performing all the malicious activities listed in \Cref{sec:model} using Sybil nodes. A lookup operation is considered eclipsed if all the peers received by the searcher consist of malicious nodes. A searcher can receive malicious peers from honest registrars (if malicious advertisers were able to place their ads) and malicious registrars (always returning the maximum amount of $N_\textit{returned}$ malicious peers). The probability of being eclipsed by a random node in a bucket thus depends on the probability of encountering a malicious registrar (determined uniquely by the number of Sybil identities) and the probability of an honest registrar returning uniquely malicious peers (determined also by the number of IP addresses under attacker's control). \Cref{fig:eclipse_probability} illustrates the probability of a lookup operation being eclipse as a function of the increasing ratio between malicious and honest nodes. For all the tested setups, the eclipse probability is close to $0$ when the attacker uses less than 4 times the amount of honest nodes participating in a topic\footnote{For comparison, a regular DHT lookup operation can be eclipsed by using a fixed amount of 20 Sybil nodes (\Cref{sec:eval})}. An attacker can eclipse $50\%$  of the lookups only when using 10 nodes per 1 honest node and providing a distinct IP address for all of them. However, such an attack would introduce a significant monetary cost to the attacker.

\begin{figure}[t]
    \includegraphics[width=1\linewidth]{img/eclipse_probability}
%    % \vspace{-2mm}
    \caption{\emph{Lookup eclipse probability}.
    }
    \label{fig:eclipse_probability}
\end{figure}

\para{\sysname achieves high resistance against DoS attacks.}
\sysname implements an admission control mechanism protecting the ad cache from being overwhelmed by an attacker with a limited number of IP addresses. Including a non-deterministic component in the ad placement mechanism reduces the efficiency of DoS attacks targeting a specific topic or a part of the hash space. Preventing honest nodes from using the system requires involving resources significantly surpassing the combined resources (see above) of the honest participants and incurs a preventive monetary cost. Importantly, all the malicious ads are removed after ad lifetime $a$. An attacker thus has to constantly use their resource to perform the attack and the system quickly recovers once the attack stops.