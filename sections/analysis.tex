\section{Analysis}\label{sec:analysis}

\subsection{Efficiency}

\para{\sysname memory usage is bounded by the capacity of the ad cache} 
The amount of ads in the cache is given by $d = xa/(a + w(x))$, where $x$ is the number of requests constantly trying to get into the table, $a$ is ad lifetime and $w(x)$ is the average waiting time received by requests $x$.
In the worst case scenario, when requests $x$ are able to achieve 0 similarity score for both the topic and the IP addresses, the waiting time formula is given by: $w(x) = ba/(1 - \frac{d}{n})^{P_\textit{occupancy}}$.

The possibility of the cache going above the capacity is determined by the constant $b$ and the exponent $P_\textit{occupancy}$. $b$ should be set to a small value to limit its influence on the waiting time (where IP and topic similarity scores should play the dominant role). $P_\textit{occupancy}$ should be large enough to prevent overflowing of the cache and small enough to enable usage of large portions of the cache under normal traffic conditions. 

In consultation with Geth developers, we assume the capacity of the cache $n = 1000$ and an average size of an advertisement equal to 1KB. \Cref{fig:cache_size_limit} presents the cache usage for different rate of incoming requests. We chose $b=10^{-7}$ and $P_\textit{occupancy}=10$ that provide protections against cache overflowing and good usage of cache space under normal conditions. 

\begin{figure}[t]
    \includegraphics[width=1\linewidth]{img/cache_size_limit}
    % \vspace{-2mm}
    \caption{Hello
    }
    \label{fig:cache_size_limit}
\end{figure}


\subsection{Fairness}


\subsection{Security}