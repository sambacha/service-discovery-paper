\section{Analysis}\label{sec:analysis}
In this section, we analyse properties of \sysname and choose optimal values for system parameters. Due to space constraints, we include an extended version of the analysis in \Cref{sec:appendix}.
\subsection{Efficiency}

\para{Memory usage is bounded by the capacity of the ad cache} 
We focus uniquely on registrars, as advertisers and searchers require a fixed amount of memory for their operations. The amount of ads in the cache is given by $d = xa/(a + w(x))$, where $x$ is the number of requests constantly trying to get into the table, $a$ is ad lifetime and $w(x)$ is the average waiting time received by requests $x$.
In the worst case scenario, when requests $x$ are able to achieve 0 similarity score for both the topic and the IP addresses, the waiting time formula is given by: $w(x) = ba/(1 - \frac{d}{n})^{P_\textit{occupancy}}$.

The possibility of the cache going above the capacity is determined by the constant $b$ and the exponent $P_\textit{occupancy}$. $b$ should be set to a small value to limit its influence on the waiting time (where IP and topic similarity scores should play the dominant role). $P_\textit{occupancy}$ should be large enough to prevent overflowing of the cache and small enough to enable usage of large portions of the cache under normal traffic conditions. 

In consultation with Geth developers, we assume the capacity of the cache $n = 1000$ and an average size of an advertisement equal to 1KB. \Cref{fig:cache_size_limit} presents the cache usage for different rate of incoming requests. We chose $b=10^{-7}$ and $P_\textit{occupancy}=10$ that provide protections against cache overflowing and good usage of cache space under normal conditions. 

\begin{figure}[t]
    \includegraphics[width=1\linewidth]{img/cache_size_limit}
    % \vspace{-2mm}
    \caption{Ad cache space usage for different request rates.
    }
    \label{fig:cache_size_limit}
\end{figure}
The pending requests (\ie not in the cache) do not create any state (apart from the lower bound) at the registrar (\ie the registrar uniquely calculates the waiting time and return a signed ticket). The lower bound state create by registrars is bounded by the number of distinct IPs and topic in the cash and is thus bounded by the capacity of the cache $O(n)$.

\para{Register and lookup operations finish within $O(log(N))$ steps}
In spite of some nondeterministic behaviour, both operations tightly follow the number of steps of the regular DHT FIND\_NODE operations and thus finish within $O(log(N))$ steps.

\subsection{Fairness}
We assume a Zipf distribution of the topics in the system and that the topic hashes are uniformly distributed in the DHT hash space. For simplicity we uniquely compare the load of \emph{registrar A} - located close to the most popular \emph{topic A} and the load of \emph{registrar B} - located close to the least popular \emph{topic B}. \emph{topic A} is followed by $N_a$ nodes and \emph{topic B} is followed by $N_b$ nodes. 

We assume \emph{topic A} and \emph{topic B} located on the opposite sides of the DHT hashspace (the worst case scenario for load balancing). Both \emph{registrar A} and \emph{registrar B} receive different amounts of traffic for both \emph{topics A} and {B} ($L_a$ and $L_b$), and the same amount of traffic $L_x$ from other topics \emph{topic X}, $\textit{topic X} \neq \textit{topic A} \land  \textit{topic X} \neq \textit{topic B}$. 

\para{Registration operations achieve equal load distribution}
As the closest node to the \emph{topic A} hash, \emph{registrar A} receives registration requests from all the $N_a$ advertisers. As the furthest node to the \emph{topic B} hash, it also receives $\frac{(N_{b}K_\textit{register})}{(\frac{N}{2})}$ requests for \emph{topic B}. Analogically, \emph{registrar B} receives $N_b$ requests for \emph{topic B} and $\frac{(N_{a}K_\textit{register})}{(\frac{N}{2})}$ requests for \emph{topic A}. As $N_a \gg N_b$, the initial number of request is higher for \emph{registrar A}. However, as ad cache fills up, \emph{registrar A} will issue higher waiting time making the requests more frequent. \Cref{fig:fairness_registration} presents the load ratio between both registrars as a function of increasing popularity between the two topics. The load difference experiences sub-linear growth. When \emph{topic A} is 100 times more popular, \emph{registrar A} receives only 1.6 times more requests than \emph{registrar B}. We present results without the admission control (\ie no waiting function) for reference\footnote{The sub-linear load ratio growth without admission control is caused by the fixed common traffic received by both registrars}. 

\begin{figure}[t]
    \includegraphics[width=1\linewidth]{img/fairness_registration}
%    % \vspace{-2mm}
    \caption{Load ratio between registrars located close to the most popular (\emph{registrar A}) and the least popular (\emph{registrar A}) topics.
    }
    \label{fig:fairness_registration}
\end{figure}

\para{Lookup operations achieve equal load distribution}
Let us assume again \emph{registrar A} being the closest node to \emph{topic A}. All the \emph{topic A} searchers $N_a$ will go towards this node during their lookup operations, so the number of requests is expected to grow as the number of $N_a$ grows. At the same time, the more participants in \emph{topic A} the more ads will be placed in other buckets further away from the \emph{registrar A}. Recall that a searchers stop its lookup operation after collecting $N_{lookup}$ peers. As the number of ads it the network grows, the more searchers are likely to stop before reaching \emph{registrar A}. \Cref{fig:fairness_lookup} presents \emph{Registrar A}'s load for increasing $N_a$. Depending on the $K_{register}$ and $K_{lookup}$ parameters, the maximum load is experienced when the number of \emph{topic A} advertisers/searchers is relatively small and goes back to $0$, when the topic becomes popular in the network. 

\begin{figure}[t]
    \includegraphics[width=1\linewidth]{img/fairness_lookup}
%    % \vspace{-2mm}
    \caption{\emph{Registrar A}'s lookup load as a function increasing popularity of \emph{topic A}.
    }
    \label{fig:fairness_lookup}
\end{figure}

\subsection{Security}

\para{Lookup operations introduce an equal load on the registrars even with unequal popularity of the topics.}
A lookup operation is considered as eclipsed if all the peers received by the searcher consist of malicious nodes. A searcher can receive malicious peer from an honest registrar (if malicious advertisers where able to place their ads) and malicious registrars (always returning the maximum amount of $N_\textit{returned}$ malicious peers. The probability of being eclipsed by a random node in a bucket thus depends on the probability of encountering a malicious registrar (determined uniquely by the number of Sybil identities) and the probability of an honest registrar returning uniquely malicious peers (determined also by the number of IP addresses under attacker's control). \Cref{fig:eclipse_analysis} illustrated the probability of a lookup operation being eclipse as a function of 

%Assuming a uniform distribution of attacker nodes\footnote{In \Cref{sec:eval} that it is the most efficient strategy for an attacker.} the probability of encountering a malicious registrar is given by $P_{malicious} = \frac{n_{malicious}}{n}$ and the probability of encountering an honest registrar $P_{honest} = 1 - P_{malicious}$. The probability of being eclipse by a single node is given by a random node in bucket $i$ is given by $P_\textit{random eclipse i} = P_{malicious} + P_{honest} * P_\textit{h eclipse i}$. The probability of being eclipsed in a bucket after asking $K_\textit{lookup}$ registrars is given by: $P_\textit{eclipse i} = {P_\textit{random eclipse i}}^{K_\textit{lookup}}$

%We need: 
%* probability of receiving uniquely malicious ads from an honest registrar in bucket `i`: %`P_h_eclipse_i`
%* probability of encountering a malicious node in a specific bucket (e.g., it's gonna be %`n_malicious/n` for the uniform distribution of the malicious nodes) `P_a_i` (`P_h_i = 1 - P_a_i`)

%With this, we can calculate the probability of receiving uniquely malicious ads from a random node in bucket `i`. This is given by `P_random_eclipse_i = P_a_i + P_h_i * P_h_eclipse_i`

%The probability of being eclipsed in a bucket after asking `K_lookup` registrars is given by: `P_eclipse_i = P_random_eclipse_i ^ K_lookup`

%Finally, the probability of being entirely eclipsed is given by: `P_eclipse = sum_i:0-17(P_eclipse_i)`


%To calculate `P_h_eclipse_i` we require:
%* The number of honest advertiser per honest registrars in bucket `i`, `n_honest_i` - this is given by `min(n_honest, (n_honest*K_register)/(n/2^i))`
%* The number of malcious advertiser per honest registrars in bucket `i`,  `n_malicious_i` - this is given by `min(n_malicious, (n_malicious*K_register)/(n/2^i))`, we can set the K_register to higher values
%* The number of IP addresses that the malicious users have `n_malicious_ip_i`
%* The number of ads in the table unrelated to the topic `d_x_i` - cab it be fixed for simplicity?