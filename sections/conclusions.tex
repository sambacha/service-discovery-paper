%!TEX root = ../main.tex

\section{Conclusions}
\label{sec:con}
On the foundational level \sysname is the first practical, secure and efficient service discovery protocol that can be deployed in large, real-world P2P networks. It combines the efficiency of traditional DHT operations with security inherited from pseudo-random ad placement. Our novel admission protocol, while performing only simple mathematical calculations, protects against a wide range of malicious behaviours, ensures equal load distribution and promotes diversity in the network.
\sysname is scheduled for deployment in the future versions of the Ethereum platform. 
An interesting future direction is to add Sybil identities detection mechanism~\cite{cholez2010efficient} and automatically modify systems parameters to operate in a more secure, but more costly, mode (\eg by decreasing the maximum number of ads retrieved from a single registrar). 

\er{minor: some issues to fix in the biblio, such as inconsistent use of acronyms, dates, or URLs. (can do)}

{\footnotesize

\smallskip

\noindent
\textbf{Open science:} All the code associated with this submission will be released open source, as well as datasets and scripts allowing to reproduce our experiments.

\smallskip

\footnotesize
\noindent
\textbf{Ethics/Prevention of harm:} This research does not introduce novel potential for harm beyond mentioning what has been published by other authors in the past~\cite{marcus2018low,henningsen2019eclipsing}.
}