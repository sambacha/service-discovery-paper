\section{Challenges}
\label{sec:challenges}

In this section, we discuss the objectives of \sysname design and the challenges involved in achieving them. A comprehensive list of objectives are listed in \Cref{tab:objectives}, and we discuss them under three main categories: security, load-balancing, and efficiency. 

\para{Security} As an open, decentralised system, \sysname must operate under a Byzantine environment. In such a setting, the malicious nodes can generate Sybils at no cost and launch various attacks on \sysname either to inflict damage on a target (\eg competition) application or to make financial gain. One example is an eclipse attack~\footnote{Eclipsing attacks can target DHT-level or application-level connections; we consider only the latter in this work.} where the application-level connections of specific peers are targeted by malicious nodes (see \Cref{sec:connections}). In addition, the malicious nodes can ignore registration requests (\ie DoS attack), return bogus results to lookup queries, or spam other registrars with ad registration requests. The latter is a misuse of the limited ad storage resources of registrars that is shared by all the advertisers across all topics.  

Once successfully registered at a registrar, an ad has a limited lifetime and is removed from the storage space upon expiration. While expiration allows new ads to be continuously registered in the network, Sybils can continuously place new ads in the network at no cost. A typical approach to avoid such misuse of shared resources is to make consumption of them costly by the users by, for instance, incorporating a Proof-of-Work (PoW) puzzle scheme \hl{[]} into the registration process---\eg registrars requiring advertisers to present PoW (\ie solution) before they are allowed to place their ads. However, PoW schemes, although widely-used in decentralised systems, unnecessarily consume resources (\textbf{-G5}), promote pooling of resources to create centralised hubs, and unfairly give advantage to resource-rich nodes ~\cite{gervais2014bitcoin}.

%Instead of a PoW scheme, we propose a lightweight \textit{waiting-time-based admission mechanism} whereby registrars enforce a waiting time on the advertisers before their ads are admitted to the topic table. Imposing this waiting time prevents misuse of the topic table, because any topic must be important enough to outweigh the cost of waiting for the advertisers, and registrars can control the rate of ad placement. 

In addition to misuse of shared ad storage space, malicious Sybils can strategically place themselves in the DHT to disrupt lookups or return bogus results (\ie a list of other Sybil nodes), for instance, as part of an eclipsing attack. The impact of malicious registrars on the discovery system depends largely on how advertisers and searchers looking for ads meet at common registrars. Since every node may act as an advertisement medium for any topic, the main challenge for nodes is to find the ``right'' subset of registrars to send advertisements and topic search queries so that they quickly meet at common nodes. At the same time, limiting the search and registration to a small subset of registrars for efficient lookups open the possibility of Sybils strategically placing themselves within the subset. Furthermore, limiting search and registrations to a small subset of nodes also results in an unfair distribution of load across registrars in the network. We discuss the fairness in load-balancing across registrars and efficiency of lookups next.

\para{Load-balancing and Efficiency} It is important that the overhead of ad registration for topics is distributed evenly across the registrars in the network while the search for ads are efficient. The overhead distribution depends largely on how the ads are distributed by the advertisers across the registrars in the network. We discuss two common solutions to distribute key, value pairs in DHTs which both turn out to be not appropriate for achieving the objectives. 

The first solution is the Kademlia's default \emph{put} and \emph{get} operations that would store all the ads for topics  on the nodes whose IDs are closest to the hash of the topic. Although this approach can uniformly distribute the topics across the network and make efficient use of routing in a structured network (\textbf{+G4, +G5}), it also results in an unequal load across registrars, especially when the popularity of the topics vary significantly. In particular, the registrars storing popular topics (\ie the ones closest to the hash of the popular topics) would receive a large portion of the registration requests in the network (\textbf{-G3}). Furthermore, in case topic-specific registrars all leave the network, the registration process would have to re-start from scratch causing disturbance in the network (\textbf{-G7}). It is also fairly easy for an attacker to generate Sybil nodes with IDs close to the topic hash and take control over the entire topic-specific traffic (\textbf{-G8}). Incremental solutions have been  proposed to enhance regular \emph{put} and \emph{get} operations by simultaneously using multiple hash functions [\hl{REF}] or increasing the number of nodes storing values for each key[\hl{REF}]. 
Unfortunately, such approaches only slightly increase the amount of resources a malicious player needs to launch a successful attack against a topic. 

The second solution is for  advertisers to place their ads on random advertisers across the entire network as done by the current discovery system of Ethereum. This approach is much more difficult to attack as a malicious player would need to take control over the entire network to control a single topic (\textbf{+G8}). Furthermore, random placement is resistant to network dynamics, as registrations are stored on multiple registrars (\textbf{+G7}) and achieves good load balance across registrars regardless of the topic popularity distribution (\textbf{+G3}). 
On the other hand, random placement makes it difficult for searchers to find placed ads, especially for unpopular topics. The main goal of structured placement (through original Kademlia DHT) is to provide bounde lookup times with good performance for large networks and good scalability. On the other hand, placing registrations in a random way does not provide any of these performance guarantees: either the advertisers must place a large number of ads to simplify the search, or searchers need to consult with potentially a large number of registrars before finding a relevant ad to simplify the registration. 
Both approaches require significant amounts of time ((\textbf{-G3}) and additional traffic (\textbf{-G4}). 
