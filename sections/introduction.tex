%!TEX root = ../main.tex
%=========================================================

\section{Introduction}

Ethereum~\cite{} is a decentralized, open-source blockchain with smart contract functionality, allowing the users to develop various decentralized applications (DApps),  and the most actively used blockchain~\cite{bloomberg} nowadays.
Following its decentralized design and structure, Ethereum relies on a communication infrastructure provided by a peer-to-peer (P2P) network, where individual and independent nodes, running an Ethereum client software (e.g., Go Ethereum~\cite{go-ethereum}), send and receive messages  containing transactions and blocks to achieve distributed consensus.

The set of network protocols which form the Ethereum peer-to-peer network are called DEVp2p. DEVp2p isn't specific to a particular blockchain, but should serve the needs of any networked application associated with the Ethereum umbrella.
Other protocols can also run on top of DEVp2p, such as 
The Whisper protocol~\cite{} (for decentralized
applications) and the Swarm protocol~\cite{} (for decentralized file
storage).
DEVp2p provides peers connection management and also node discovery services.
Between the DEVp2p network protocols and the app protocol (i.e., Ethereum, Whisper, Swarm, etc), RLPx~\cite{} provides a secure transport layer. 

%DEVp2p  manages
%connections to other peers which form the overlay on which
%blocks and transactions are distributed.
%Go-ethereum client by default has
%a total of 25 TCP connections to other peers speaking the
%Ethereum protocol. Of these 25 slots, 17 are reserved for
%inbound connections (initiated by other peers), whereas the remaining 8 are allocated for outbound connections. 
%No further restrictions
%apply to inbound connections; if an inbound slot is available
%Go-ethereum client simply accepts any connecting peer that supports the
%Ethereum protocol and operates on the same network (main,
%testing, etc.). 
%The 8 outbound slots are therefore especially
%important, as they are the most difficult ones to get under
%control for an attacker mounting an eclipse attack.
%In contrast to DEVp2p, the discv4 node discovery stores
%information about all node types in the overlay. This includes
%nodes without support for the Ethereum protocol (which is a
%perfectly valid case in the design logic of Ethereum’s protocol
%stack). The discv4 node discovery is inspired by the Kademlia
%DHT [7]. Information about known overlay nodes is stored in
%a table separated into so-called k-buckets (or simply buckets,
%in the following).
%
%A key part of DEVp2p is the discovery protocol. 
%
%
%
%discv4 is a Kademlia-inspired system but doesn't store any files, only node information is relayed. We chose Kademlia primarily because the algorithm is simple and understandable while providing a distributed database that scales with the number of participants. Our system also relies on the routing table to allow enumeration and random traversal of the whole network, i.e. all participants can be found. Most importantly, having a structured network with routing enables thinking about DHT 'address space' and 'regions of address space'. These concepts are used to build the topic-based node index.
%
%Kademlia is often criticized as a naive design with obvious weaknesses. We believe that most issues with simple Kademlia can be overcome by careful programming and the benefits of a simple design outweigh the cost and risks of maintaining a more complex system.



%Why discovery service is key for a p2p network.
%Devp2p intro
%Motivation of Discv5
%Partitioned p2p network for earch service vs single p2p network with service discovery

The new Ethereum node discovery version 5 (Discv5) is a system and the associated protocols for finding participants in a peer-to-peer (P2P) network. Different from the previous version of the discovery protocol (version 4), the new discovery system allows nodes to associate themselves with a set of tags, i.e., topics, and enable searching of peers that advertise themselves as associated with particular topics. Topics can be used to identify different things such as an application or a service, a certain functionality of a node, or any other attribute by which the node may wish to belong to a connected service-specific “subnetwork” of the Ethereum network. In this document, we focus on the application of Discv5 for the discovery of service-specific peers.



-

It is envisioned that the Discv5 system will be used across many different services at large scale, mainly to form subnetworks consisting of the service peers. The basic objectives of Node Discovery v5 is to provide a facility for nodes to:
\begin{itemize}
    \item register 'topic advertisements', i.e., “ads”, at other peers in the Ethereum network
    \item find other nodes that advertised a particular topic by querying selected peers in the Ethereum network.
\end{itemize}

    
    
