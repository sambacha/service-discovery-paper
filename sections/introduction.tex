%!TEX root = ../main.tex
%=========================================================

\section{Introduction}

The new Ethereum node discovery version 5 (Discv5) is a system and the associated protocols for finding participants in a peer-to-peer (P2P) network. Different from the previous version of the discovery protocol (version 4), the new discovery system allows nodes to associate themselves with a set of tags, i.e., topics, and enable searching of peers that advertise themselves as associated with particular topics. Topics can be used to identify different things such as an application or a service, a certain functionality of a node, or any other attribute by which the node may wish to belong to a connected service-specific “subnetwork” of the Ethereum network. In this document, we focus on the application of Discv5 for the discovery of service-specific peers.

It is envisioned that the Discv5 system will be used across many different services at large scale, mainly to form subnetworks consisting of the service peers. The basic objectives of Node Discovery v5 is to provide a facility for nodes to:
\begin{itemize}
    \item register 'topic advertisements', i.e., “ads”, at other peers in the Ethereum network
    \item find other nodes that advertised a particular topic by querying selected peers in the Ethereum network.
\end{itemize}

    
    
