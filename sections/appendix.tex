%!TEX root = ../main.tex

\newpage
\appendix
\section*{Appendix: Extended Analysis}
\label{sec:appendix}

In this appendix, we provide more details on the analysis described in Section~\ref{sec:analysis}.

\subsection{Efficiency}

Due to the admission procedure, an advertisement for a specific topic will stay only $\frac{a}{a+w}$ of the time in the table of a registrar, where $a$ is the ad lifetime in the table (when admitted) and $w$ is the average waiting time. Assuming the ``worst case'' scenario where a request achieves 0 similarity score for both the topic and its IP address, the waiting time is $w = \frac{b\cdot a}{(1 - \frac{d}{n})^{P_\textit{occ}}}$, where $n$ is the table capacity and $b$ and $P_{occ}$ are constants defined in Section~\ref{sec:waitingTime}.
If a registrar receives $x$ such requests for that topic, the average number $d$ of ads in the table is therefore
\begin{equation}
d = \frac{x}{1 + b \cdot (1 - \frac{d}{n})^{-P_\textit{occ}}} \label{eq:efficiency_d}
\end{equation}
To obtain the results shown in Figure~\ref{fig:cache_size_limit}, Equation~\ref{eq:efficiency_d} must be numerically solved for $d$. With increasing $x$, $d$ will asymptotically approach the table capacity $n$.

\subsection{Fairness}

\subsubsection{Load distribution of registration operations}

We consider two topics $A$ and $B$ that are located on the opposite sides of the DHT hashspace. The registrar closest to topic $A$ will receive ad registration requests from all $N_a$ advertisers of topic $A$. On the other hand, from the viewpoint of the $N_b$ advertisers of topic $B$, the same registrar is located in the furthest  and largest bucket, containing half of the $N$ network nodes. Therefore, the registrar will only receive requests from, on average, $\frac{N_b\cdot K_\textit{register}}{N/2}$ advertisers for topic $B$.

Extending Equation~\ref{eq:efficiency_d} by the topic similarity score and still assuming the ``worst case'' of complete IP address diversity, the average number of ads $d_a$ and $d_b$ for topic $A$ and $B$, respectively, in the registrar's table will be:
\begin{eqnarray}
d_a & = & \frac{N_a}{1 + (b + \frac{d_a}{d}) \cdot (1 - \frac{d}{n})^{-P_\textit{occ}}}\label{eq:da_fair}\\
d_b & = & \frac{\frac{N_b\cdot K_\textit{register}}{N/2}}{1 + (b + \frac{d_b}{d}) \cdot (1 - \frac{d}{n})^{-P_\textit{occ}}}\\
\label{eq:da_fair2}
d_x & = & \frac{N_x}{1 + (b + \frac{d_x}{d}) \cdot (1 - \frac{d}{n})^{-P_\textit{occ}}}
\end{eqnarray}
where $d = d_a + d_b + d_x$. $N_X$ is a ``background load'' representing the other topics in the system.

Again, the above system of non-linear equations must be solved numerically for $d_a$ and $d_b$.
If topic $A$ is far more popular than topic $B$, i.e., $N_a \gg N_b$, $d_a$ will be greater than $d_b$. However, because of the topic similarity score $\frac{d_a}{d}$, the advertisements for topic $A$ will not occupy the entire table.

For a registrar the closest to topic $B$ and the furthest from topic $A$, the same equations, but with the roles of topics $A$ and $B$ switched, are obtained. We assume that both registrars will experience the same background load $N_x$.

\subsubsection{Load distribution of lookup operations}

As described in Section~\ref{sec:lookup}, a searcher looking for topic $A$ will start at the furthest bucket and progress toward the closest registrar to that topic until it has received $N_\textit{lookup}$ responses.
Let bucket~1 be the furthest bucket containing $N/2$ registrars, bucket~2 the closer bucket containing $N/4$ registrars etc. In each bucket $i$, the searcher will query $\max(K_\textit{lookup}, \frac{N}{2^i})$ registrars.
The lookup analysis in Section~\ref{sec:analysis} requires to calculate the probability that a searcher will reach the last bucket, i.e., the closest registrar to the topic. In the following, we calculate the distribution $p_{1..t}(Resp=S)$ of the number of responses $S$ that a searcher will have received after traversing buckets 1 to $t$.

Let $p_{rr,i}(Resp=S)$ be the probability that the searcher will receive $S$ responses from a registrar $rr$ in bucket $i$. We have
\begin{eqnarray*}
p_{rr,i}(Resp=S) & = & \sum_{R=0}^{N_a}{} p( \mbox{$rr$ received $R$ registrations}\\
                        & \wedge &  \mbox{$rr$ has $\min(S,N_\textit{return})$ ads})
\end{eqnarray*}
where $N_\textit{return}$ is the maximum number of responses a registrar will return. The probability that the registrar received $R$ registrations is the probability that $R$ of the $N_a$ advertisers chose the registrar. It is given by the binomial distribution:
\begin{eqnarray*}
p(\mbox{$rr$ received $R$ registrations}) = \\
\binom{N_a}{R} \left( \frac{K_\textit{register}}{N/2^i} \right)^R \left(1 - \frac{K_\textit{register}}{N/2^i} \right)^{N_a-R}
\end{eqnarray*}
Given $R$ registrations, the number of ads that the registrar returns can be calculated using Equation~\ref{eq:da_fair} by substituting $N_a$ with $R$. Combining both results, we can calculate the joint probability $p_{rr,i}(Resp=S)$.

Calculating the exact probability $p_{i}(Resp=S)$ that a searcher obtains $S$ responses in bucket $i$ and the probability $p_{1..t}(Resp=S)$ to obtain $S$ responses after visiting buckets 1 to $t$ is numerically intensive.
Instead, we use Monte-Carlo simulation to approximate $p_{1..t}(Resp=S)$ from $p_{rr,i}(Resp=S)$. For each bucket $1\le i \le t$, the simulation randomly draws $K_\textit{lookup}$ samples from the distribution $p_{rr,i}(Resp=S)$, in this way simulating the querying of $K_\textit{lookup}$ registrars per bucket. This approximation assumes that the distributon of responses for the individual registrars in a bucket are independent, which is mostly correct for large $N$. The simulation is repeated 100,000 times for the result shown in Figure~\ref{fig:fairness_lookup}.
The probability to reach the last registrar is then $p_{1..u}(Resp< N_\textit{lookup})$, where $u$ is the number of buckets.

% TODO:  explain calculation of overlaps

\subsection{Security}

The probability that a searcher is eclipsed when looking up a topic in $t$ buckets is the probability to only receive malicious ads from the contacted registrars. The probability that a searcher will contact a malicious registrar in a bucket $i$ is $P_{m,i}$ .
The probability $P_i$ to receive only malicious ads from a random registrar in bucket $i$ is therefore
$$ P_{i} = P_{m,i} + (1-P_{m,i}) \cdot P_{h,i}$$
where $P_{h,i}$ is the probability that a honest registrar in bucket $i$ only returns malicious ads.
Assuming independence, the probability to only receive malicious ads from the entire bucket $i$ is $P_{i}^{K_\textit{lookup}}$ and the probability $P_{1..t}$ to be eclipsed in all $t$ visited buckets is then
$$ P_{1..t} = \prod_{i=1}^{t} P_{i}^{K_\textit{lookup}}$$

\subsubsection{Finding $P_{m,i}$}

The probability $P_{m,i}$ that an individual registrar in bucket $i$ is malicious depends on what source the searcher used to add nodes to that bucket.
If the searcher itself is located in bucket $u$, it can be expected that, on average, the buckets $1$ to $u$ are fully populated by the searcher from its own routing table and that the routing table is also able to provide 17 uniformly distributed nodes for the remaining buckets. Since the buckets get smaller and smaller, we can except that we will get 8 nodes for bucket $u+1$, 4 nodes for bucket $u+2$, 2 nodes for bucket $u+3$, and one node for bucket $u+4$ from the routing table.
The probability for such a node to be malicious is $\frac{n_a}{N}$, where $n_a$ is the number of malicious nodes in the network.

This means that, for $K_\textit{register}=K_\textit{lookup}=5$, the searcher has to use the nodes returned by the queried registrars to fill the empty slots in buckets $i>u+1$, as explained in Section~\ref{sec:struct}. If the queried registrar is honest, it will return nodes that are malicious with probability $\frac{n_a}{N}$. However, if the registrar is malicious, all the nodes it returns are likely malicious, too. Consequently, the probability that a node in bucket $i>u+1$ is malicious will be greater than $\frac{n_a}{N}$ and will increase with $i$.

For the results shown in Figure~\ref{fig:eclipse_probability}, we calculate $P_{m,i}$ using Monte-Carlo simulation and assume that $u=2$ which is the expected bucket for a random searcher.


\subsubsection{Finding $P_{h,i}$}

To calculate $P_{h,i}$, we need to know the distribution of honest and malicious ads in a honest registrar.
If the registrar contains $d_h$ honest and $d_a$ malicious ads, $P_{hi}$ is the probability to randomly select only malicious ads from those $d_h+d_a$ ads, which is $\prod_{i=0}^{d_h+d_a} \frac{d_a - i}{d_h+d_a-i}$.

Calculating the distribution of $d_h$ and $d_a$ analytically is difficult due to the complexity of the waiting time calculation. Instead, we approximate $P_{h,i}$ using the averages of $d_h$ and $d_a$.
On average, a registrar in bucket $i$ will receive $R_h = \frac{N_h\cdot K_\textit{register}}{N/2^i}$ honest and $R_a = \frac{N_a\cdot K'_\textit{register}}{N/2^i}$ malicious registration requests, where $N_h$ and $N_a$ are the number of honest and malicious registrars and $K_\textit{register}$ and $K'_\textit{register}$ their respective number of registration attempts.

The computation of the average $d_h$ and $d_a$ is, in principle, similar to Equations~\ref{eq:da_fair} and Equations~\ref{eq:da_fair2}. However, since an attacker will only have a limited number of IP addresses, we cannot ignore anymore the impact of the IP similarity score on the waiting time:
\begin{eqnarray}
d_h & = & \frac{R_h}{1 + (b + \frac{d_h}{d} + \textit{IPscoreH}) \cdot (1 - \frac{d}{n})^{-P_\textit{occ}}}\\
d_a & = & \frac{R_a}{1 + (b + \frac{d_a}{d} + \textit{IPscoreA}) \cdot (1 - \frac{d}{n})^{-P_\textit{occ}}}
\end{eqnarray}
where $d = d_h + d_a$. Obviously, the IP score for honest ads (\textit{IPscoreH}) and malicious ads (\textit{IPscoreA}) depends on the numbers of honest ads and malicious ads in the table and on the number of different IP addresses used by the advertisers. Again, these non-linear equations must be solved numerically.

In the next section, we will explain how to calculate averages for the IP scores. To simplify the understanding we will start with simple situations and complete them step by step.

\subsection{IP Score}
\textbf{Situation 1: There are already} \(\mathbf{n}\) \textbf{random
addresses in the tree. What score will a new random IP address get?}

\emph{Note: We assume completely random addresses. That means it can
happen with a certain probability that some of the random addresses are
identical.}

First level of the tree: With probability 0.5, the new address goes to the 0-branch
and the probability that branch contains more than \(\frac{n}{2}\) of
the entries is
\(P(\# 0 \geq \left\lfloor \frac{n}{2} \right\rfloor + 1)\). Also with
probability 0.5, the new address goes to the 1-branch and the
probability that branch contains more than \(\frac{n}{2}\) of the
entries is\(\ P(\# 1 \geq \left\lfloor \frac{n}{2} \right\rfloor + 1).\)
Since the binomial distribution is symmetric,
\(P\left( \# 0 \geq \left\lfloor \frac{n}{2} \right\rfloor + 1 \right) = \ P(\# 1 \geq \left\lfloor \frac{n}{2} \right\rfloor + 1)\).
The average score for the first level is:

\[\text{NatScore}_{\text{level}}(n) = \sum_{i = \left\lfloor \frac{n}{2} \right\rfloor + 1}^{n}\frac{\left( \frac{n}{i} \right)}{2^{n}} = 1 - \sum_{i = 0}^{\left\lfloor \frac{n}{2} \right\rfloor}\frac{\left( \frac{n}{i} \right)}{2^{n}}\]

%This will introduce an error if n is not an integer and multiple of two.
%The best seems to be to take the average of the two approximations:
%\[\text{NatScor}e_{\text{level}}(n) = \frac{1}{2}\left( 1 - \sum_{i = 0}^{\left\lfloor \frac{n}{2} \right\rfloor}\frac{\left( \frac{n}{i} \right)}{2^{n}} + \sum_{i = \left\lfloor \frac{n}{2} \right\rfloor + 1}^{n}\frac{\left( \frac{n}{i} \right)}{2^{n}} \right)\ \ \ \ \ \ \ \ \ \ \ \ \ \ \ \ \ \ \ \ \ \ (*)\]

We assume that the levels are independent and that we will have, on
average, \(\frac{n}{2^{i - 1}}\) entries in each subtree of level \(i\).
Using the above equation on each level gives

\[\text{Score}_{\text{random}}(n) = \frac{1}{32} \sum_{i = 1}^{32}{\text{NatScore}_{\text{level}}\left( \frac{n}{2^{i - 1}} \right)}\]

\textbf{Situation 2: There are already} \(\mathbf{n + k}\)
\textbf{addresses in the tree, however} \(\mathbf{n}\) \textbf{are
random and} \(\mathbf{k}\) \textbf{are identical. What score will a new
random IP address get?}

\emph{Note: We assume again completely random addresses. That means it can
happen with a certain probability that one of the random addresses is
identical to another random address or even to the} \(k\)
\emph{identical ones.}

Let's first define the probability that at least \(\text{b\ }\)out of
\(m\) fair coin tosses are head:

\[p(m,b) = \sum_{i = b}^{m}\frac{\left( \frac{m}{i} \right)}{2^{m}}\]

%Again, errors will be introduced if \(m\) is not a natural number.
%Better results are obtained with the approximation in equation (*) from
%situation 1.

First level (``level 0'') of the tree: With probability 0.5, the new IP address is
in the same branch as the \(k\) identical addresses. The probability
that more than half of the entries are in this branch knowing that
the $k$ identical are in this branch is:

\[p\left( n,\ \left\lfloor \frac{n + k}{2} \right\rfloor + 1 - k \right)\]

With probability 0.5, the new IP address is in the branch that only
contains random addresses. The probability to have the majority of the
entries in this branch knowing that \(k\) are definitely not in this
branch is:

\[p\left( n,\ \left\lfloor \frac{n + k}{2} \right\rfloor + 1 \right)\]

In total, the score for the first level is

\[\frac{1}{2}p\left( n,\ \left\lfloor \frac{n + k}{2} \right\rfloor + 1 - k \right) + \frac{1}{2}p\left( n,\ \left\lfloor \frac{n + k}{2} \right\rfloor + 1 \right)\]

On the second level, we have four branches, of which one contains with
certainty the \(k\) identical addresses:

\[\frac{1}{4}p\left( \frac{n}{2},\ \left\lfloor \frac{n + k}{4} \right\rfloor + 1 - k \right) + \frac{3}{4}p\left( \frac{n}{2},\ \left\lfloor \frac{n + k}{4} \right\rfloor + 1 \right)\]

And so on for the other levels. In total, the average score is:

\[
\begin{split}
\text{Score}_{\text{random}}(n,k) = \frac{1}{32} \sum_{i = 1}^{32}\left\lbrack \frac{1}{2^{i}}p\left( \frac{n}{2^{i - 1}},\ \left\lfloor \frac{n + k}{2^{i}} \right\rfloor + 1 - k \right)\right. \\
+ \left.\left( 1 - \frac{1}{2^{i}} \right) \cdot p\left( \frac{n}{2^{i - 1}},\ \left\lfloor \frac{n + k}{2^{i}} \right\rfloor + 1 \right) \right\rbrack
\end{split}
\]



\textbf{Situation 3: There are already} \(\mathbf{n + k}\)
\textbf{addresses in the tree, however} \(\mathbf{n}\) \textbf{are
random and} \(\mathbf{k}\) \textbf{are identical (also random). What
score will an IP address get that is identical to the} \(\mathbf{k}\)
\textbf{identical entries?}

The situation is similar to situation 2. However, we know that we always
stay on the branch with the \(k\) identical entries. The score is:

\[\text{Score}_{\text{identical}}(n,k) = \frac{1}{32} \sum_{i = 1}^{32}{p\left( \frac{n}{2^{i - 1}},\ \left\lfloor \frac{n + k}{2^{i}} \right\rfloor + 1 - k \right)}\]

\textbf{Situation 4: There are already}
\(\mathbf{n +}\mathbf{2}^{\mathbf{a}}\mathbf{\cdot k}\)
\textbf{addresses in the tree, however} \(\mathbf{n}\) \textbf{are
random and there are} \(\mathbf{2}^{\mathbf{a}}\) \textbf{``groups'' of}
\(\mathbf{k}\) \textbf{entries with identical addresses. Those}
\(\mathbf{2}^{\mathbf{a}}\) \textbf{addresses are distributed perfectly
over the tree. What score will a new random IP address get?}

Since the \(2^{a}\) addresses are distributed perfectly over the tree,
the new random address will see exactly \(2^{a - 1} \cdot k\) of them in
its branch at the first level, \(2^{a - 2} \cdot k\) in its branch at
the second level etc. After level \(a\), the subtrees start to behave
like in situation 2.

The score for the level \(1 \leq j \leq a\) is

\[p\left( \frac{n}{2^{j - 1}},\ \left\lfloor \frac{n + 2^{a}k}{2^{j}} \right\rfloor + 1 - 2^{a - j}k \right)\]

After level \(a\), we can apply the score of situation 2 to each
subtree:

\[
\begin{split}
{\text{IPScoreH}(n,k) = \frac{1}{32}\sum_{j = 1}^{a}{p\left( \frac{n}{2^{j - 1}},\ \left\lfloor \frac{n + 2^{a}k}{2^{j}} \right\rfloor + 1 - 2^{a - j}k \right)}
}\\
+ \frac{1}{32}\sum_{i = a + 1}^{32}\left\lbrack \frac{1}{2^{i - a}}p\left( \frac{n}{2^{i - 1}},\ \left\lfloor \frac{n + 2^{a}k}{2^{i}} \right\rfloor + 1 - k \right)+\right.\\
\left.\left( 1 - \frac{1}{2^{i - a}} \right) \cdot p\left( \frac{n}{2^{i - 1}},\ \left\lfloor \frac{n + 2^{a}k}{2^{i}} \right\rfloor + 1 \right) \right\rbrack
\end{split}
\]

\textbf{Situation 5: Like situation 4
(}\(\mathbf{n +}\mathbf{2}^{\mathbf{a}}\mathbf{\cdot k}\)
\textbf{addresses), but the new address is not random; it is one of the}
\(\mathbf{2}^{\mathbf{a}}\) \textbf{addresses.}

The score for levels 1 to \(a\) is the same as in situation 4. However,
for the levels \(> a\), the ``new'' address will always stay on a branch
that contains the \(k\) entries with the same address. For those levels,
we can use the result from situation 3.

\[
\begin{split}
\text{IPScoreA}(n,k) = \frac{1}{32} \sum_{j = 1}^{a}{p\left( \frac{n}{2^{j - 1}},\ \left\lfloor \frac{n + 2^{a}k}{2^{j}} \right\rfloor + 1 - 2^{a - j}k \right)}\\
+ \frac{1}{32} \sum_{i = a + 1}^{32}{p\left( \frac{n}{2^{i - 1}},\ \left\lfloor \frac{n + 2^{a}k}{2^{i}} \right\rfloor + 1 - k \right)}
\end{split}
\]

