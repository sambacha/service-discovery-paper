\section{System model}

The topic advertisement subsystem indexes participants by their registered topic identifiers or topic(s), for short. A node providing a certain service, each identified with a topic, is said to 'place an ad' for itself when it registers an ad on another peer to make itself discoverable under that topic. Depending on the needs of the application, a node can advertise multiple topics or no topics at all. Every node participating in the discovery protocol acts as an 'advertisement medium', meaning that it accepts topic ads from other nodes and later returns them to nodes searching for the same topic, keeping an extra topic table (in addition to the Kademlia neighbours table) tracking their neighbors by topic index.

\begin{itemize}
    \item A 'topic' is an identifier for a service provided by a node.
    \item An 'advertiser' is a node providing a service that wants to be discovered.
    \item An 'ad' (i.e., advertisement) is the registration of an advertiser for a topic on another node.
    \item A 'registrar' is a node that is acting as the advertisement medium in a service protocol interaction, i.e., a node that is contacted to store an ad belonging to an advertiser.
    \item A 'searcher' is a node looking for ads for a topic.

\end{itemize}