%!TEX root = ../main.tex
%=========================================================

\section{System and threat models}
\label{sec:model}

\begin{table*}
    \caption{Notation}
    \label{tab:notation}
\begin{center}
    \begin{tabular}{ | l | p{16cm} |}
      \hline
        $N$ & network size \hl{Consider changing to all nodes so that $n \in N$} \\ \hline
        $N_a$ & number of nodes (advertisers) following topic $a$ \hl{not very consistent with the rest of the notation}\\ \hline
        $N_h$ & number of honest registrars (Appendix) \hl{clash}. \\ \hline
        $N_a$ & number of malicious registrars (Appendix) \hl{clash}. \\ \hline
        $k$ & number of peers holding a key in the DHT (from the background section) \hl{not sure we need it - it's only in the background and we can explain everything by finding just 1 closest peer to the key} \\ \hline
        $X$ & a node (from the background section) - \hl{never used again} \\ \hline
        $x$ & registration request (Sec. 7.1) \hl{used uniquely in 7.1} \\ \hline
        $x$ & number of received requests (Appendix) \hl{clash} \\ \hline
        $n$ & \hl{a node (from the 4.1)} \\ \hline
        $n$ & total size of the ad cache (Sec. 5.2) \hl{a clash with node} \\ \hline
        $n$ & (Appendix) number of random addresses in the cache \hl{clash }\\ \hline
        $k$ & (Appendix) number of identical addresses in the cache \hl{clash }\\ \hline
        $n_a$ & number of malicious nodes in the network (Appendix) \hl{clash} \\ \hline
        $p$ & peer (section 4.1) \\ \hline
        $p_{\geq i}$ & (sec. 6) the number of IP addresses in the cache sharing a prefix with $IP$ with a length of at least $i$ \hl{clash with peer}. \\ \hline
        $T$ & Table (section 4.1) \\ \hline
        $b_i$& $i$-th bucket (section 4.1) \hl{TODO: add key} \\ \hline
        $\alpha$ & number of concurrent routing operations \\ \hline
        $K_{\textit{register}}$ & number of ads placed per bucket\\ \hline
        $K'_{\textit{register}}$ & $K_{\textit{register}}$ for malicious nodes (Appendix) \hl {not very clear}\\ \hline
        $K_{\textit{lookup}}$ & \hl{It's weirdly defined in Sec. 4.4}: search process progresses using maximum $K_{\textit{lookup}}$  parallel requests, and issuing requests towards up to $K_{\textit{lookup}}$  peers per bucket.\\ \hline
        $a$ & ad lifetime  (Sec. 4.3)\\ \hline
        $N_{\textit{lookup}}$ & number of peers to find (Sec. 4.4) - \hl{could be introduced before specifying what the system does}  \hl{Clash with number of nodes} \\ \hline
        $N_{\textit{return}}$ & max number of topic-specific peers returned from a single registrar \hl{Clash with number of nodes} \\ \hline
        $t_{\textit{init}}$ & ticket: the local time at the registrar when the ad was received for the first time \\ \hline
        $t_{\textit{waiting}}$ & ticket the local time at the registrar when the ad was received for the first time \\ \hline
        $\delta t_{\textit{window}}$ & registration window \hl{we probably don't need both $\delta$ and "window"} \\ \hline
        $t_{\textit{cumulative}}$ & ticket:cumulative waiting time \\ \hline
        $t_1$ & a specific moment in time \\ \hline
        $w$ & waiting time (Sec. 5.3) \\ \hline
        $w(\textit{IP}, \textit{topic})$ & waiting time for IP and topic \\ \hline
        $w(t_i)$ & waiting time at time $t_1$ \hl{clash with the above} \\ \hline
        $w(x)$ & average waiting time received by request $x$ (Sec. 7.1.) \hl{clash with the above} \\ \hline
        $w$ & is the average waiting time (Appendix) \hl{clash} \\ \hline
        $r$ & number of registration retries after which the advertiser give up \hl{it's not in the algorithm} \\ \hline
        $d$ & number of ads in the cache \\ \hline
        $d(\textit{topic})$ & number of ads for topic in the cache \\ \hline
        $d_a$ & the average number of ads for topic $a$ (appendix) \hl{clash with the above} \\ \hline
        $d_h$ & number of honest ads in the cache (appendix) \hl{clash with the above} \\ \hline
        $d_a$ & number of malicious ads in the cache (appendix) \hl{clash with the above} \\ \hline
        $\textit{similarity score}$ & similarity score \\ \hline
        $\textit{similarity}$ & similarity score \\ \hline
        $\textit{similarity(topic)}$ & topic similarity score \\ \hline
        $\textit{similarity(IP)}$ & IP similarity score \\ \hline
        $\textit{score(IP)}$ & IP score (before normalisation) \\ \hline
        $\textit{penalty}(p_{\geq i})$ & penalty points at a specific level \\ \hline
        $P_\textit{occ}$ & occupancy power \\ \hline
        $b$ & system parameter from the similarity score \hl{clash with the bucket} \\ \hline
        $\textit{IPs}$ & all the IPs in the ad cache \\ \hline
        $\textit{topics}$ & all the topics in the ad cache \\ \hline
        $w_\textit{topic}$ & topic specific part of the lower bound \hl{not really defined in text}\\ \hline
        $\textit{bound(topic)}$ & lower bound for a topic \hl{not really defined in text}\\ \hline
        $\textit{timestamp(topic)}$ & time when the topic-specific bound was set \hl{not really defined in text}\\ \hline
        $\textit{registrar A}$ & registrar $A$ (Sec. 7.2)\\ \hline
        $\textit{topic A}$& topic $A$ (Sec. 7.2)\\ \hline
        $p_{rr,i}(Resp=S)$ & (Appendix) the probability that the searcher will receive $S$ responses from a registrar $rr$ in bucket $i$. \\ \hline
        $rr$ & registrar $rr$ (Appendix) \\ \hline
        $Resp = S$ &  number of received responses equals $S$ (Appendix) \\ \hline
        $R$ & (Appendix) number of received registrations \\ \hline
        $R_h$ & (Appendix) number of received honest registrations \\ \hline
        $R_a$ & (Appendix) number of received malicious registrations \\ \hline
        $u$ & (Appendix) the number of buckets \\ \hline
        $t$ & the number of buckets a searcher looks in (Appendix) \hl{clash with time}\\ \hline
        $P_i$ & (Appendix) probability to receive only malicious ads from a random registrar in bucket $i$ \\ \hline
        $P_{m, i}$ & (Appendix) probability that a searcher will contact a malicious registrar in bucket $i$ \\ \hline
        $P_{h, i}$ & (Appendix) is the probability that a honest registrar in bucket $i$ only returns malicious ads. \\ \hline
        $\textit{IPscoreH}$ & (Appendix) the IP score of an honest IP \\ \hline
        $\textit{IPscoreA}$ & (Appendix) the IP score of a malicious IP \\ \hline
        $\text{NatScore}_{\text{level}}(n)$ & the average score for the first level (Appendix) \\ \hline
        $\text{Score}_{\text{random}}(n)$ & the average similarity score received for a random honest IP (Appendix) \\ \hline
        $p(m,b) $ & (Appendix) the probability that $b$ out of $m$ fair coin tosses are head \\ \hline
        $\text{Score}_{\text{random}}(n,k)$ & score will a new random IP address get assuming $n$ random IPs and $k$ identical ones \\ \hline
        $\text{Score}_{\text{identical}}(n,k)$ & score will a new identical IP address get assuming $n$ random IPs and $k$ identical ones \\ \hline
        $a$ & (Appendix) exponent for the number of identical IP groups in the tree \hl{clash}\\ \hline
        $\text{IPScoreH}(n,k)$ & (Appendix) the score of a new honest IP address assuming $n$ random IPs and $2^a \times k$ identical ones \\ \hline
        $\text{IPScoreA}(n,k)$ & (Appendix) the score of a new malicious IP address assuming $n$ random IPs and $2^a \times k$ identical ones \\ \hline
      \hline
    \end{tabular}
  \end{center}
\end{table*}
\mk{TODO: add notation for number of malicious nodes, number of malicious IPs etc.,}
\mk{change bucket definition to add the target key (can be useful to define advertise tables etc. if we introduce them in the background \eg $b_i(key)$). We also often indicate the farthest buckets - we could simply use $b_0$}
\mk{Put macros for the notation}
\mk{We need a notation for an ad - at the end of Sec. 5.2 we just use "Ad".}
\mk{We should probably define notations for registrars, advertisers, searchers etc. We can then easily explain that everyone is a registrar etc.}
\mk{Maybe define an eclipse ratio for a specific topic?}
\mk{If we define malicious nodes and their IPs early, we don't have to re-do it in the evaluation}



\begin{table*}
  \caption{New Notation}
  \label{tab:new_notation}
\begin{center}
  \begin{tabular}{ | l | p{16cm} |}
    \hline
      $n \in N$ & nodes in the network \\ \hline
      $|N|$ & number of nodes in the network \\ \hline
      $n^h \in N^h$ & honest nodes in the network \\ \hline
      $n^m \in N^m$ & malicious nodes in the network \\ \hline
      $a^h(x) \in A^h(x)$ & honest advertisers for topic $x$ \\ \hline
      $r^h(x) \in R^h(x)$ & honest registrars for topic $x$ \\ \hline
      $g_i \in G$  & group, topic, key, app, pack\\ \hline
      $d_{t1}^h(x)$& number of honest ads for topic $x$ in the ad cache at time $t_1$ \\ \hline
      $e$ & expiry time (ad lifetime) \\ \hline
      $b_i(x) \in B(x)$& $i$ bucket centred around key $x$, in routing table centred around $x$ \\ \hline
      $\algvar{topic}_i$\\ \hline
      $w_{t1}(x) $ & waiting time of for request $x$ at time $t1$ \\ \hline

    \hline
  \end{tabular}
\end{center}
\end{table*}

In this section, we present our network and threat models as well as the target properties of \sysname. 

\subsection{System model}

We assume a network of $N$ nodes.\footnote{$N$ is unknown to the participants but is used in our analysis.}
At startup time, each node generates a public/secret key pair, which it uses to secure point-to-point communication with its peers.
Nodes are identified by their \emph{node ID}, which is simply the hash of their public key.
Multiple nodes may share the same IP address (due to NAT or being hosted by the same physical machine)~\cite{marcus2018low}.
However, two nodes cannot share the same ID.

% \er{revise: It is not clear \sysname is an extension of discv5 from the online documentation, but rather its central mechanism. What are the other protocols as part of discv5  is also unclear.}
% \sysname is built on the existing Ethereum DHT. Specifically, we designed and implemented our system as an extension of Ethereum's \emph{Node Discovery Protocol v5 (discv5)}. However, the operation of service discovery is not Ethereum-specific and could also be implemented using a different DHT. \er{I would not necessarily make this claim here, since several operations are. Kademlia-specific in reality. Maybe save for the conclusion?}

As for \discv, \sysname leverages the existing Ethereum DHT, but does not rely on its key lookup operations.
\sysname indexes different applications in the Ethereum eco-system as \emph{topics}.
A topic is an identifier for an application, or \emph{service}, provided by a node.
Topics are arbitrary strings that hash to a specific key. % in the DHT key space.

All DHT nodes fulfill the following roles:
\begin{itemize}
    \item A \textbf{registrar} accepts advertisements (\emph{ads}) and responds to topic queries. 
    When asked for a specific topic, a registrar responds with nodes that registered ads for this topic.
    % All DHT nodes act as registrars. 
    A registrar may accept and store ads for any topic.
    \item An \textbf{advertiser} registers for a specific service topic and wants to be discovered by its peers.
    Advertisers make themselves discoverable by placing ads for that topic.
    Nodes are advertisers for every service they participate in.
    \item A \textbf{searcher} attempts to discover nodes registered under a specific topic, using service lookup operations.
\end{itemize}

A node providing a certain service topic is said to \emph{register} itself when it submits an ad to a registrar. % to make itself discoverable under that topic.
Depending on the needs of the application, a node can advertise multiple topics or no topics at all. 
We assume that the popularity of topics in the system is highly heterogeneous, i.e., it can follow a power law distribution~\cite{kim2018measuring}.
Anyone can participate in registering and searching for (one or more) topics and use the same ID and IP for all its topics. 

\subsection{Threat Model}
\label{sec:threat}

\sysname is designed to operate in an open, adversarial environment.
We assume the presence of malicious actors in the network that may arbitrarily deviate from the protocol and coordinate their actions.
Malicious actors can spawn multiple virtual nodes within one physical machine, and operate multiple physical machines.
As a result, they can control a number of Sybil node IDs.

% \er{should we mention collusions here? It is mentioned only at the end of the section (although we do not seem to consider it so much in the experiments)}\mk{In the experiments we assume that all the Sybils ae controlled by a single entity. I've expanded the phrase above. I hope it's fine now.}
% % These actors do not strictly follow the protocol when communicating with others and attempt to influence the discovery results of honest nodes by steering them toward the attacker-controlled nodes.

The security of the service discovery mechanism is fundamentally dependent on the security guarantees provided by the underlying DHT implementation, and in particular on its ability to avoid eclipse attacks against a specific node.
Indeed, even if \sysname does not rely on DHT lookup operations, information in the DHT routing table is used to initialize and maintain the data structures used by the service discovery layer.
We assume, therefore, that no honest node is fully eclipsed by the malicious ones, \ie, each honest DHT node has \emph{at least} one honest peer.
As previously mentioned, Ethereum already implements multiple mechanisms preventing eclipse attacks at the DHT level~\cite{marcus2018low, henningsen2019eclipsing}.  

Maintaining nodes in the DHT requires infrastructure resources (public IP addresses) and we assume that the resources under the control of an attacker are limited.
Specifically, we assume that it is easier for an attacker to generate similar IP addresses (\ie within a single subnet) than it is to control many diverse IP addresses (with different prefixes).
We use the number of IP addresses and IDs under the control of an attacker as parameters for our evaluation in \Cref{sec:eval}.

\mk{We can probably remove those and mention those problems when introducing our system.}
Through a literature review~\cite{chen2020survey, henningsen2019eclipsing} we collected a list of malicious behaviors that an attacker can use to disrupt a DHT-based service discovery protocol:
\begin{itemize}
    \item \textbf{Malicious DHT Peer}: DHT routing relies on asking peers for other nodes, closer to a specific target. When responding to such requests, an attacker only returns other malicious nodes, in order to hijack the process of DHT traversal by honest nodes. \er{this seems contradictory with the fact that we wrote two paragraphs ago that we assumed the DHT to be secured against Sybils. Should we change DHT routing to something more generic, like an exploration process?} \er{also, this may appear contradictory to the fact we do not use the DHT routing (lookup) directly but implement our own}
    \item \textbf{Malicious Registrar}: When queried for a specific topic, an attacker returns a maximum amount of malicious nodes.\footnote{Alternatively, a malicious registrar could simply refuse to respond. However, such behavior is less effective than returning malicious peers.}
    \item \textbf{Spamming Advertisers}: An attacker bombards honest registrars with malicious ads. The attack can target single or multiple topics. The attack may cause an honest registrar to refuse honest ads due to lack of resources (\eg, storage and CPU power). This registrar will, furthermore, later return maliciously-inserted ads of the spammer when queried by honest searchers. 
    \item \textbf{Generic Spammers}: An attacker bombards an honest registrar with topic queries or random traffic hoping to exhaust bandwidth or CPU power.
\end{itemize}

Adversaries can strategically target specific nodes or regions in the DHT key space (\ie by generating Sybil Node IDs within the desired region) when implementing these attacks.
We assume the presence of attackers that can combine some or all of the behaviors above, and coordinate their actions to maximize their effect. 

\subsection{Target Properties}

Under the considered threat model, \sysname achieves the following properties.

\para{Efficiency} 
\sysname ensures that the number of nodes contacted and the number of messages exchanged increase logarithmically with the number of system participants, for both lookup and register operations. 
Sending and processing service discovery requests requires only simple operations involving a constant amount of resources.
The storage usage for the registrars is limited by a configurable but fixed cap that does not depend on the amount of incoming traffic. 

\para{Fairness}
\sysname ensures a balanced load distribution across systems participants and regions of the key space (\ie, it avoids \emph{hotspots}).
The system provides efficient lookup and registration operations for all participants regardless of the topic they look up or register for.
Each advertiser has a similar probability of being discovered by its peers. 
% \er{we should ideally be a bit more formal here and give bounds (in big O?) of the unbalance between these nodes.}\mk{What fairness metric would you suggest here?} \er{this would be ratio between the likelihood of being discovered for a popular and a less-popular advertiser}

\para{Security}
\sysname focuses specifically on preventing \emph{Denial of Service} (nodes are unable/slow to discover their application-specific peers) and \emph{Eclipse} (nodes discover uniquely malicious nodes) attacks.
While completely eliminating such attacks in an open network is technically impossible, \sysname provides high probabilistic security 
% \er{can we quantify this probability?}\mk{It depends on a lot of factors \#Sybil, \#malicious\_IPs etc. so not easy, but I'll think about it.} \er{commenting for now, no time}
and increases the monetary cost of committing them. 
% under the presence of a powerful attacker.
